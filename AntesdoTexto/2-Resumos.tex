\onehalfspacing  % Sets line spacing to 1.5
\begin{center}
    \textbf{RESUMO}
\end{center}

{\noindent Este relatório apresenta uma análise experimental do comportamento transiente de um sistema de condicionamento de ar automotivo sob diferentes condições de operação. Foram medidos temperatura, pressão, umidade relativa e vazão mássica em pontos estratégicos do sistema durante cinco ensaios com perturbações provocadas pela variação da rotação do compressor e da vazão de ar do ventilador do evaporador. Observou-se que o aumento da vazão de ar gera resposta oscilatória, com sobressinal de 2 °C e 3 °C para as temperaturas de sucção e descarga, respectivamente, ao aumentar a vazão mássica de 0,037 kg/s para 0,071 kg/s, enquanto alterações na rotação do compressor resultam em resposta suave, sem sobressinal. As maiores histereses ocorreram na descarga do compressor, atingindo 16\% na forma normalizada. Quanto ao desempenho, o COP aumentou de 26\%  com aumento da vazão de ar e diminuiu 33\% com sua redução. Foi identificada histerese de aproximadamente 10\% no COP. Ele também reduziu durante o aumento da rotação do compressor e cresceu  durante sua redução, diminuindo 30\% no aumento da rotação de 500 RPM para 890 RPM e aumentando 13\% na diminuição da rotação do compressor de 1074 RPM para 890 RPM.
\\\\
\noindent \textbf{Palavras-chave}: Coeficiente de desempenho (COP); Comportamento transiente; Condicionamento de ar automotivo; Histerese. 
}
\newpage

\begin{center}
    \textbf{ABSTRACT}
\end{center}

{\noindent This report presents an experimental analysis of the transient behavior of an automotive air conditioning system under different operating conditions. Temperature, pressure, relative humidity, and mass flow rate were measured at strategic points of the system during five tests with disturbances caused by variations in compressor speed and evaporator fan airflow. It was observed that increasing airflow generates an oscillatory response, with overshoots of 2 °C and 3 °C for suction and discharge temperatures, respectively, when the mass flow rate increases from 0.037 kg/s to 0.071 kg/s, while changes in compressor speed result in a smooth response without overshoot. The largest hysteresis occurred at the compressor discharge, reaching 16\% in normalized form. Regarding performance, the COP increased by 26\% with higher airflow and decreased by 33\% with lower airflow. A hysteresis of approximately 10\% was identified in the COP. It also decreased during compressor speed increase and rose during its reduction, dropping by 30\% when the speed increased from 500 RPM to 890 RPM, and rising by 13\% when the compressor speed decreased from 1074 RPM to 890 RPM.
\\\\
\noindent \textbf{Keywords}: Automotive air conditioning; Coefficient of performance (COP); Hysteresis; Transient behavior.
}

\onehalfspacing
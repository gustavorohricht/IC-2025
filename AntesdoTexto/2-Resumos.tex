\onehalfspacing  % Sets line spacing to 1.5
\begin{center}
    \textbf{RESUMO}
\end{center}

{\noindent Este relatório apresenta uma análise experimental do comportamento transiente de um sistema de condicionamento de ar automotivo sob diferentes condições de operação. Foram medidos temperatura, pressão, umidade relativa e vazão mássica em pontos estratégicos do sistema durante cinco ensaios com perturbações provocadas pela variação da rotação do compressor e da vazão de ar do ventilador do evaporador. Observou-se que o aumento da vazão de ar gera resposta oscilatória, com sobressinal de aproximadamente 6\%, enquanto alterações na rotação do compressor resultam em resposta suave, sem sobressinal. As maiores histereses ocorreram na descarga do compressor, atingindo 16\% na forma normalizada. Quanto ao desempenho, o COP reduziu 43\% com aumento da vazão de ar e aumentou 22\% com sua redução. Foi identificada histerese de aproximadamente 11\% no COP. O COP diminuiu durante o aumento da rotação do compressor e aumentou durante sua redução, diminuindo 25\% no aumento da rotação e 500 RPM para 890 RPM e aumentando 12\% na diminuição da rotação do compressor de 1074 RPM para 890 RPM.
\\\\
\noindent \textbf{Palavras-chave}: Coeficiente de desempenho (COP); Comportamento transiente; Condicionamento de ar automotivo; Histerese. 
}
\newpage

\begin{center}
    \textbf{ABSTRACT}
\end{center}

{\noindent This report presents an experimental analysis of the transient behavior of an automotive air conditioning system under different operating conditions. Temperature, pressure, relative humidity, and mass flow rate were measured at strategic points of the system during five tests with disturbances caused by variations in compressor speed and evaporator fan airflow. It was observed that increasing airflow generated an oscillatory response, with an overshoot of approximately 6\%, while changes in compressor speed resulted in a smooth response without overshoot. The largest hysteresis occurred at the compressor discharge, reaching 16\% in normalized form. Regarding performance, the COP decreased by 43\% with increased airflow and increased by 22\% with its reduction. A hysteresis of approximately 11\% was identified in the COP. The COP decreased during the increase in compressor speed and increased during its reduction, decreasing by 25\% when increasing speed from 500 RPM to 890 RPM and increasing by 12\% when reducing speed from 1074 RPM to 890 RPM.
\\\\
\noindent \textbf{Keywords}: Automotive air conditioning; Coefficient of performance (COP); Hysteresis; Transient behavior.
}

\onehalfspacing
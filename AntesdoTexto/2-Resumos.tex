\begin{center}
    \textbf{RESUMO}
\end{center}

{\fontsize{12pt}{14pt}\selectfont \noindent Este relatório apresenta uma análise experimental do comportamento transiente de um sistema automotivo de condicionamento de ar automotivo considerando diferentes condições de operação. Para tanto, a partir da literatura disponível, foi elaborado o plano experimental e a adaptação do aparato experimental de condicionamento de ar automotivo do laboratório de refrigeração veicular (REVE) para coleta de dados. Durante os ensaios, serão coletados os dados das medições de temperatura, pressão, umidade relativa e vazão mássica em diferentes pontos do sistema de condicionamento de ar automotivo com o intuito de montar uma base de dados para a análise e comparação dos resultados. Desta forma, serão estimadas as incertezas experimentais e as propriedades do sistema para descrever o funcionamento dos equipamentos nos aspectos gerais, realizando o balanço de massa e energia em cada componente da bancada experimental. Os ensaios experimentais serão realizados entre as faixas de operação de 500 até 890 rpm no compressor, com temperatura controlada por dispositivos eletrônicos, mantendo a temperatura do condensador e a temperatura no evaporador constantes e alterando a vazão de ar no evaporador.
\\\\
\noindent \textbf{Palavras-chave}: Condicionamento de ar automotivo; Eficiência energética; Aquisição transiente de dados experimentais.
}
\newpage

\begin{center}
    \textbf{ABSTRACT}
\end{center}

{\fontsize{12pt}{14pt}\selectfont \noindent 
This report presents an experimental analysis of the transient behavior of an automotive air conditioning system considering different operating conditions. For this purpose, based on the available literature, an experimental plan was developed and the experimental apparatus for automotive air conditioning of the Vehicle Refrigeration Laboratory (REVE) was adapted for data collection. During the tests, data from measurements of temperature, pressure, relative humidity and mass flow will be collected at different points of the automotive air conditioning system in order to assemble a database for analysis and comparison of the results. In this way, the experimental uncertainties and the properties of the system will be estimated to describe the operation of the equipment in general aspects, performing the mass and energy balance in each component of the experimental bench. The experimental tests will be carried out between the operating ranges of 500 to 890 rpm in the compressor, with temperature controlled by electronic devices, keeping the condenser temperature and evaporator temperature constant and changing the air flow in the evaporator.
\\\\
\noindent\textbf{Keywords}: Automotive air conditioning; Energy efficiency; Transient acquisition of experimental data
}
\onehalfspacing
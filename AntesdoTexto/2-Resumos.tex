\onehalfspacing  % Sets line spacing to 1.5
\begin{center}
    \textbf{RESUMO}
\end{center}

{\noindent Este relatório apresenta uma análise experimental do comportamento transiente de um sistema de condicionamento de ar automotivo sob diferentes condições de operação. Foram medidos temperatura, pressão, umidade relativa e vazão mássica em pontos estratégicos do sistema durante cinco ensaios com perturbações provocadas pela variação da rotação do compressor e da vazão de ar do ventilador do evaporador. Observou-se que o aumento da vazão de ar gera resposta oscilatória, com sobressinal de aproximadamente 6\%, enquanto alterações na rotação do compressor resultam em resposta suave, sem sobressinal. As maiores histereses ocorreram na descarga do compressor, atingindo 16\% na forma normalizada. Quanto ao desempenho, o COP reduziu 43\% com aumento da vazão de ar e aumentou 22\% com sua redução. Foi identificada histerese no COP, com diminuição de cerca de 11\% durante a elevação da rotação do compressor e aumento de aproximadamente 12\% na redução desta.
\\\\
\noindent \textbf{Palavras-chave}: Coeficiente de desempenho (COP); Comportamento transiente; Condicionamento de ar automotivo; Histerese. 
}
\newpage

\begin{center}
    \textbf{ABSTRACT}
\end{center}

{\noindent This report presents an experimental analysis of the transient behavior of an automotive air conditioning system under different operating conditions. Temperature, pressure, relative humidity, and mass flow rate were measured at strategic points of the system during five tests with disturbances caused by variations in the compressor rotational speed and in the evaporator fan airflow rate. It was observed that increasing the airflow rate produces an oscillatory response, with an overshoot of approximately 6\%, while changes in the compressor speed result in a smooth response without overshoot. The largest hysteresis occurred at the compressor discharge, reaching 16\% in normalized form. Regarding performance, the COP decreased by 43\% with increased airflow rate and increased by 22\% with its reduction. Hysteresis in COP was also identified, with a decrease of about 11\% during the increase in compressor speed and an increase of approximately 12\% during its reduction.
\\\\
\noindent \textbf{Keywords}: Automotive air conditioning; Coefficient of performance (COP); Hysteresis; Transient behavior.
}

\onehalfspacing
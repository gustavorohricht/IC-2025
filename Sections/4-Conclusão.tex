\section{CONCLUSÃO}

Neste trabalho, foi proposta a análise experimental do comportamento transiente de um sistema automotivo de condicionamento de ar, para isso, foi realizada a identificação dos modos de operação para controle da rotação do compressor, a adaptação da bancada experimental e análise dos resultados de um sistema de condicionamento de ar automotivo do laboratório de refrigeração veicular (REVE), assim como dos modos de operação. 

O aparato experimental passou por ajustes e modificações necessárias e encontra-se plenamente preparado para a realização dos experimentos.

O plano experimental foi estruturado para garantir que as condições entre os transientes estejam definidas para a realização dos testes. Essa abordagem visa garantir a confiabilidade e a reprodutibilidade dos dados coletados durante os experimentos. O aluno que dará continuidade aos experimentos já está atuando no laboratório e em treinamento para obtenção dos dados experimentais.
    
Sendo assim, a execução das etapas previstas até o momento está concluída. Os testes previstos no cronograma do projeto serão realizados a partir do início de 2025.

\subsection{Verificar se é necessário incluir como o IC contribuiu pra minha formação}

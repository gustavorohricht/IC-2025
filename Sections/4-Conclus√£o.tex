\section{CONCLUSÃO}

Neste trabalho, foi proposta a análise experimental do comportamento transiente de um sistema automotivo de condicionamento de ar do laboratório de refrigeração veicular (REVE) e do seu ciclo de histerese. Para isso, foram realizados ensaios experimentais causando perturbações na frequência de rotação do compressor e na vazão de ar do ventilador do evaporador. Ao todo, foram realizadas cinco perturbações distintas para estudar o resultado das perturbações e da histerese no sistema de refrigeração. 

Foi possível observar que, para perturbações pela variação da vazão de ar do ventilador, isto é, para P1 e P3, o sistema apresenta um comportamento oscilatório, principalmente ao aumentar a vazão de ar, com sobressinal de aproximadamente 1 °C ou 6\% do valor final em estado estacionário. Já quando a vazão de ar do ventilador é reduzida, o sistema apresenta um comportamento mais estável, sem sobressinal, e comportamento mais similar a sistemas de primeira ordem. Para a perturbação P2, que consiste na variação da frequência de rotação do compressor, o sistema apresenta um comportamento suave, também sem sobressinal. 

As perturbações P4 e P5 também são perturbações na frequência de rotação do compressor, mas com foco na análise do ciclo de histerese. Foi notado que a histerese está presente em todas as variáveis do sistema, as maiores histereses estão na região de descarga do compressor, com a maior histerese normalizada observada sendo de 0,16 ou 16\%. Isto é, há uma diferença de 16\% entre os caminhos de subida e descida da frequência de rotação do compressor. 

No COP foi demonstrado experimentalmente que quando há aumento da vazão de ar, o COP decresce e quando a vazão de ar diminui neste sistema, o COP cresce. Foi observada uma redução de 43\% na perturbação P1 e um aumento de 22\% na perturbação P3, quando comparado ao estado estacionário. 

Para as perturbações de rotação do compressor como P2, P4 e P5 é possível observar que o COP apresenta um comportamento suave, sem sobressinal, assim como as demais variáveis do sistema em transiente. O COP diminui quando a rotação do compressor aumenta e aumenta quando a rotação do compressor diminui, o que é o oposto da capacidade de refrigeração do sistema. Para a perturbação P2, por exemplo, o COP apresentou uma diminuição de 25\% quando comparado ao estado estacionário. 

Além disso, foi possível observar expressiva histerese no COP, o principal responsável por isso sendo, provavelmente, a mudança abrupta da potência de consumo do compressor. Em P4  observou-se uma diminuição de aproximadamente 11\% no COP quando comparado ao estado estacionário, enquanto que em P5, o COP aumentou 12\% quando comparado ao estado estacionário.  O que indica que o sistema retorna, mesmo que com histerese, ao seu estado estacionário anterior após a perturbação.

Ademais, não foram observadas correlações significativas entre as histereses normalizadas do sistema e as constantes de tempo das variáveis do sistema, o que indica que a histerese observada no sistema não é diretamente proporcional às constantes de tempo do sistema. Um controle não adaptativo, no entanto, utilizando as menores constantes de tempo observadas como referência para a taxa de amostragem dos sensores e escolhendo uma banda morta em torno do valor final da variável de interesse, pode ser uma abordagem viável para lidar com a histerese observada no sistema.

Por fim, para trabalhos futuros, recomenda-se a realização de mais ensaios experimentais com diferentes condições de operação do sistema, como variações na temperatura ambiente e umidade relativa do ar, para uma análise mais abrangente do comportamento transiente e da histerese do sistema de condicionamento de ar automotivo. Além disso, a implementação de técnicas de controle não adaptativo e adaptativo, conforme mencionadas anteriormente, pode ser uma abordagem interessante para confirmar a eficácia das estratégias de controle propostas.

\subsection{Como o IC Contribuiu para a Minha Formação}

A realização deste trabalho de Iniciação Científica foi fundamental para o meu desenvolvimento acadêmico e profissional. Através da análise experimental do sistema de condicionamento de ar automotivo, pude aplicar os conhecimentos teóricos adquiridos ao longo do curso em um contexto prático, além de adquirir conhecimentos que vão além dos ensinados em sala, desenvolvendo habilidades essenciais como a coleta e análise de dados, a interpretação e tratamento de resultados experimentais e a elaboração de relatórios técnicos.  Além disso, a experiência me proporcionou uma compreensão mais profunda no que diz respeito ao comportamento transiente e à histerese em sistemas de refrigeração veicular. A interação com os equipamentos do laboratório REVE e a realização de ensaios experimentais foram enriquecedoras, permitindo-me aprimorar minhas habilidades técnicas e de resolução de problemas. Também tive a oportunidade de me aprofundar em conceitos de controle, como a análise de histerese e a influência das variáveis do sistema no desempenho do condicionamento de ar, o que é essencial para o desenvolvimento de sistemas mais eficientes e sustentáveis.
Essas experiências foram valiosas para minha formação, pois me proporcionaram uma formação mais sólida, além de despertar meu interesse por pesquisas futuras na área de refrigeração veicular e controle de sistemas. 
